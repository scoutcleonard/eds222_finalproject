% Options for packages loaded elsewhere
\PassOptionsToPackage{unicode}{hyperref}
\PassOptionsToPackage{hyphens}{url}
%
\documentclass[
]{article}
\usepackage{amsmath,amssymb}
\usepackage{lmodern}
\usepackage{ifxetex,ifluatex}
\ifnum 0\ifxetex 1\fi\ifluatex 1\fi=0 % if pdftex
  \usepackage[T1]{fontenc}
  \usepackage[utf8]{inputenc}
  \usepackage{textcomp} % provide euro and other symbols
\else % if luatex or xetex
  \usepackage{unicode-math}
  \defaultfontfeatures{Scale=MatchLowercase}
  \defaultfontfeatures[\rmfamily]{Ligatures=TeX,Scale=1}
\fi
% Use upquote if available, for straight quotes in verbatim environments
\IfFileExists{upquote.sty}{\usepackage{upquote}}{}
\IfFileExists{microtype.sty}{% use microtype if available
  \usepackage[]{microtype}
  \UseMicrotypeSet[protrusion]{basicmath} % disable protrusion for tt fonts
}{}
\makeatletter
\@ifundefined{KOMAClassName}{% if non-KOMA class
  \IfFileExists{parskip.sty}{%
    \usepackage{parskip}
  }{% else
    \setlength{\parindent}{0pt}
    \setlength{\parskip}{6pt plus 2pt minus 1pt}}
}{% if KOMA class
  \KOMAoptions{parskip=half}}
\makeatother
\usepackage{xcolor}
\IfFileExists{xurl.sty}{\usepackage{xurl}}{} % add URL line breaks if available
\IfFileExists{bookmark.sty}{\usepackage{bookmark}}{\usepackage{hyperref}}
\hypersetup{
  pdftitle={USDA National School Lunch Program Time Series Analysis},
  pdfauthor={Scout Leonard},
  hidelinks,
  pdfcreator={LaTeX via pandoc}}
\urlstyle{same} % disable monospaced font for URLs
\usepackage[margin=1in]{geometry}
\usepackage{color}
\usepackage{fancyvrb}
\newcommand{\VerbBar}{|}
\newcommand{\VERB}{\Verb[commandchars=\\\{\}]}
\DefineVerbatimEnvironment{Highlighting}{Verbatim}{commandchars=\\\{\}}
% Add ',fontsize=\small' for more characters per line
\usepackage{framed}
\definecolor{shadecolor}{RGB}{248,248,248}
\newenvironment{Shaded}{\begin{snugshade}}{\end{snugshade}}
\newcommand{\AlertTok}[1]{\textcolor[rgb]{0.94,0.16,0.16}{#1}}
\newcommand{\AnnotationTok}[1]{\textcolor[rgb]{0.56,0.35,0.01}{\textbf{\textit{#1}}}}
\newcommand{\AttributeTok}[1]{\textcolor[rgb]{0.77,0.63,0.00}{#1}}
\newcommand{\BaseNTok}[1]{\textcolor[rgb]{0.00,0.00,0.81}{#1}}
\newcommand{\BuiltInTok}[1]{#1}
\newcommand{\CharTok}[1]{\textcolor[rgb]{0.31,0.60,0.02}{#1}}
\newcommand{\CommentTok}[1]{\textcolor[rgb]{0.56,0.35,0.01}{\textit{#1}}}
\newcommand{\CommentVarTok}[1]{\textcolor[rgb]{0.56,0.35,0.01}{\textbf{\textit{#1}}}}
\newcommand{\ConstantTok}[1]{\textcolor[rgb]{0.00,0.00,0.00}{#1}}
\newcommand{\ControlFlowTok}[1]{\textcolor[rgb]{0.13,0.29,0.53}{\textbf{#1}}}
\newcommand{\DataTypeTok}[1]{\textcolor[rgb]{0.13,0.29,0.53}{#1}}
\newcommand{\DecValTok}[1]{\textcolor[rgb]{0.00,0.00,0.81}{#1}}
\newcommand{\DocumentationTok}[1]{\textcolor[rgb]{0.56,0.35,0.01}{\textbf{\textit{#1}}}}
\newcommand{\ErrorTok}[1]{\textcolor[rgb]{0.64,0.00,0.00}{\textbf{#1}}}
\newcommand{\ExtensionTok}[1]{#1}
\newcommand{\FloatTok}[1]{\textcolor[rgb]{0.00,0.00,0.81}{#1}}
\newcommand{\FunctionTok}[1]{\textcolor[rgb]{0.00,0.00,0.00}{#1}}
\newcommand{\ImportTok}[1]{#1}
\newcommand{\InformationTok}[1]{\textcolor[rgb]{0.56,0.35,0.01}{\textbf{\textit{#1}}}}
\newcommand{\KeywordTok}[1]{\textcolor[rgb]{0.13,0.29,0.53}{\textbf{#1}}}
\newcommand{\NormalTok}[1]{#1}
\newcommand{\OperatorTok}[1]{\textcolor[rgb]{0.81,0.36,0.00}{\textbf{#1}}}
\newcommand{\OtherTok}[1]{\textcolor[rgb]{0.56,0.35,0.01}{#1}}
\newcommand{\PreprocessorTok}[1]{\textcolor[rgb]{0.56,0.35,0.01}{\textit{#1}}}
\newcommand{\RegionMarkerTok}[1]{#1}
\newcommand{\SpecialCharTok}[1]{\textcolor[rgb]{0.00,0.00,0.00}{#1}}
\newcommand{\SpecialStringTok}[1]{\textcolor[rgb]{0.31,0.60,0.02}{#1}}
\newcommand{\StringTok}[1]{\textcolor[rgb]{0.31,0.60,0.02}{#1}}
\newcommand{\VariableTok}[1]{\textcolor[rgb]{0.00,0.00,0.00}{#1}}
\newcommand{\VerbatimStringTok}[1]{\textcolor[rgb]{0.31,0.60,0.02}{#1}}
\newcommand{\WarningTok}[1]{\textcolor[rgb]{0.56,0.35,0.01}{\textbf{\textit{#1}}}}
\usepackage{graphicx}
\makeatletter
\def\maxwidth{\ifdim\Gin@nat@width>\linewidth\linewidth\else\Gin@nat@width\fi}
\def\maxheight{\ifdim\Gin@nat@height>\textheight\textheight\else\Gin@nat@height\fi}
\makeatother
% Scale images if necessary, so that they will not overflow the page
% margins by default, and it is still possible to overwrite the defaults
% using explicit options in \includegraphics[width, height, ...]{}
\setkeys{Gin}{width=\maxwidth,height=\maxheight,keepaspectratio}
% Set default figure placement to htbp
\makeatletter
\def\fps@figure{htbp}
\makeatother
\setlength{\emergencystretch}{3em} % prevent overfull lines
\providecommand{\tightlist}{%
  \setlength{\itemsep}{0pt}\setlength{\parskip}{0pt}}
\setcounter{secnumdepth}{-\maxdimen} % remove section numbering
\ifluatex
  \usepackage{selnolig}  % disable illegal ligatures
\fi

\title{USDA National School Lunch Program Time Series Analysis}
\author{Scout Leonard}
\date{11/16/2021}

\begin{document}
\maketitle

\begin{Shaded}
\begin{Highlighting}[]
\FunctionTok{library}\NormalTok{(here)}
\FunctionTok{library}\NormalTok{(lubridate)}
\FunctionTok{library}\NormalTok{(tidyverse)}
\FunctionTok{library}\NormalTok{(zoo)}
\FunctionTok{library}\NormalTok{(feasts)}
\FunctionTok{library}\NormalTok{(tsibble)}

\FunctionTok{options}\NormalTok{(}\AttributeTok{scipen =} \DecValTok{999}\NormalTok{)}
\end{Highlighting}
\end{Shaded}

\hypertarget{my-question}{%
\section{My Question:}\label{my-question}}

My analysis seeks to explore the questions: Is there seasonality in how
U.S. school food programs feed students? What about long-term trends,
and how did any trends change in 2020 during school closures from the
Covid-19 pandemic?

\hypertarget{background}{%
\subsection{Background}\label{background}}

The National School Lunch Program (NSLP) is an enormous food system with
major implications for equity in American K-12 education systems. Every
day, NSLP provides \textasciitilde30 million children school lunch at
free or reduced prices (1). It operates in public and nonprofit private
schools and residential childcare facilities (2).

To provide meals at free and reduced cost to students, participating
school districts are reimbursed cash subsidies for every qualifying meal
they serve. To qualify for subsidy, meals served by Nutrition Services
operators must meet federal meal pattern policies which define meal
content around qualifying food group combinations, sugar content, etc.

\hypertarget{data-description}{%
\section{Data Description}\label{data-description}}

\begin{Shaded}
\begin{Highlighting}[]
\CommentTok{\#read in the monthly lunch data }
\NormalTok{usda\_monthly }\OtherTok{\textless{}{-}} \FunctionTok{read.csv}\NormalTok{(}\FunctionTok{here}\NormalTok{(}\StringTok{"data"}\NormalTok{, }\StringTok{"usda\_monthly\_data\_tidy.csv"}\NormalTok{))}
\end{Highlighting}
\end{Shaded}

\begin{Shaded}
\begin{Highlighting}[]
\CommentTok{\#change first column name }
\FunctionTok{colnames}\NormalTok{(usda\_monthly)[}\DecValTok{1}\NormalTok{] }\OtherTok{\textless{}{-}} \StringTok{"month"}

\CommentTok{\#delete the weird columns that got added between downloading the raw data to my }
\CommentTok{\#local computer and reading it to .Rmd}
\NormalTok{usda\_monthly }\OtherTok{\textless{}{-}} \FunctionTok{select}\NormalTok{(usda\_monthly, }\SpecialCharTok{{-}}\FunctionTok{c}\NormalTok{(}\StringTok{"X"}\NormalTok{, }\StringTok{"X.1"}\NormalTok{))}

\CommentTok{\#change last column name}
\FunctionTok{colnames}\NormalTok{(usda\_monthly)[}\DecValTok{9}\NormalTok{] }\OtherTok{\textless{}{-}} \StringTok{"fiscal\_year"}
\end{Highlighting}
\end{Shaded}

\begin{Shaded}
\begin{Highlighting}[]
\CommentTok{\#remove percent signs from percent\_free\_of\_total\_lunches and }
\CommentTok{\#percent\_reduced\_price\_of\_total\_lunches columns}
\NormalTok{usda\_monthly }\OtherTok{\textless{}{-}}\NormalTok{ usda\_monthly }\SpecialCharTok{\%\textgreater{}\%} 
 \FunctionTok{mutate}\NormalTok{(}\AttributeTok{percent\_free\_of\_total\_lunches =} \FunctionTok{gsub}\NormalTok{(}\StringTok{\textquotesingle{}\%\textquotesingle{}}\NormalTok{,}\StringTok{\textquotesingle{}\textquotesingle{}}\NormalTok{, percent\_free\_of\_total\_lunches)) }\SpecialCharTok{\%\textgreater{}\%} 
  \FunctionTok{mutate}\NormalTok{(}\AttributeTok{percent\_reduced\_price\_of\_total\_lunches =} \FunctionTok{gsub}\NormalTok{(}\StringTok{\textquotesingle{}\%\textquotesingle{}}\NormalTok{,}\StringTok{\textquotesingle{}\textquotesingle{}}\NormalTok{, percent\_reduced\_price\_of\_total\_lunches)) }\SpecialCharTok{\%\textgreater{}\%} 
  \FunctionTok{mutate}\NormalTok{(}\AttributeTok{month =} \FunctionTok{gsub}\NormalTok{(}\StringTok{"{-}"}\NormalTok{, }\StringTok{" "}\NormalTok{, month))}

\CommentTok{\#convert month column to class datetime from class character}
\NormalTok{usda\_monthly }\OtherTok{\textless{}{-}}\NormalTok{ usda\_monthly }\SpecialCharTok{\%\textgreater{}\%} 
  \FunctionTok{mutate}\NormalTok{(}\StringTok{"month"} \OtherTok{=}\NormalTok{ zoo}\SpecialCharTok{::}\FunctionTok{as.yearmon}\NormalTok{(month, }\StringTok{"\%y \%b"}\NormalTok{))}
\end{Highlighting}
\end{Shaded}

\begin{Shaded}
\begin{Highlighting}[]
\CommentTok{\#check class of all columns}
\FunctionTok{lapply}\NormalTok{(usda\_monthly, class)}
\end{Highlighting}
\end{Shaded}

\begin{verbatim}
## $month
## [1] "yearmon"
## 
## $total_participation
## [1] "integer"
## 
## $total_lunches_served
## [1] "integer"
## 
## $percent_free_of_total_lunches
## [1] "character"
## 
## $percent_reduced_price_of_total_lunches
## [1] "character"
## 
## $snacks_served
## [1] "integer"
## 
## $cash_payments
## [1] "integer"
## 
## $total_commodity_costs
## [1] "integer"
## 
## $fiscal_year
## [1] "integer"
\end{verbatim}

\begin{Shaded}
\begin{Highlighting}[]
\CommentTok{\#apply numeric to character columns}
\NormalTok{usda\_monthly }\OtherTok{\textless{}{-}}\NormalTok{ usda\_monthly }\SpecialCharTok{\%\textgreater{}\%} 
  \FunctionTok{mutate\_if}\NormalTok{(is.character, as.numeric)}

\CommentTok{\#check to see if class conversion worked }
\FunctionTok{lapply}\NormalTok{(usda\_monthly, class)}
\end{Highlighting}
\end{Shaded}

\begin{verbatim}
## $month
## [1] "yearmon"
## 
## $total_participation
## [1] "integer"
## 
## $total_lunches_served
## [1] "integer"
## 
## $percent_free_of_total_lunches
## [1] "numeric"
## 
## $percent_reduced_price_of_total_lunches
## [1] "numeric"
## 
## $snacks_served
## [1] "integer"
## 
## $cash_payments
## [1] "integer"
## 
## $total_commodity_costs
## [1] "integer"
## 
## $fiscal_year
## [1] "integer"
\end{verbatim}

\begin{Shaded}
\begin{Highlighting}[]
\CommentTok{\#plots }
\NormalTok{total\_lunches\_initial }\OtherTok{\textless{}{-}} \FunctionTok{ggplot}\NormalTok{(}\AttributeTok{data =}\NormalTok{ usda\_monthly, }\FunctionTok{aes}\NormalTok{(}\AttributeTok{x =}\NormalTok{ month, }\AttributeTok{y =}\NormalTok{ total\_lunches\_served)) }\SpecialCharTok{+}
  \FunctionTok{geom\_line}\NormalTok{() }\SpecialCharTok{+}
  \FunctionTok{theme\_minimal}\NormalTok{()}

\NormalTok{total\_lunches\_initial}
\end{Highlighting}
\end{Shaded}

\includegraphics{blog_draft_files/figure-latex/unnamed-chunk-6-1.pdf}

\begin{Shaded}
\begin{Highlighting}[]
\NormalTok{percent\_free\_inital }\OtherTok{\textless{}{-}} \FunctionTok{ggplot}\NormalTok{(}\AttributeTok{data =}\NormalTok{ usda\_monthly, }\FunctionTok{aes}\NormalTok{(}\AttributeTok{x =}\NormalTok{ month, }\AttributeTok{y =}\NormalTok{ percent\_free\_of\_total\_lunches)) }\SpecialCharTok{+}
  \FunctionTok{geom\_line}\NormalTok{() }\SpecialCharTok{+}
  \FunctionTok{theme\_minimal}\NormalTok{()}

\NormalTok{percent\_free\_inital}
\end{Highlighting}
\end{Shaded}

\includegraphics{blog_draft_files/figure-latex/unnamed-chunk-7-1.pdf}

\hypertarget{analysis-plan}{%
\section{Analysis Plan}\label{analysis-plan}}

\hypertarget{total-participation-in-nslp}{%
\subsection{Total Participation in
NSLP}\label{total-participation-in-nslp}}

\begin{Shaded}
\begin{Highlighting}[]
\CommentTok{\#confusingly, converting the data I want in my time series requires data of class \textasciigrave{}yearmonth\textasciigrave{} not \textasciigrave{}yearnon\textasciigrave{} I learned I can use \textasciigrave{}yearmonth()\textasciigrave{} from the tsibble package to make a happy tsibble}
\NormalTok{usda\_monthly }\OtherTok{\textless{}{-}}\NormalTok{ usda\_monthly }\SpecialCharTok{\%\textgreater{}\%} 
  \FunctionTok{mutate}\NormalTok{(}\AttributeTok{month\_tsib =} \FunctionTok{yearmonth}\NormalTok{(month))}

\NormalTok{monthly\_tsib }\OtherTok{\textless{}{-}}\NormalTok{ usda\_monthly }\SpecialCharTok{\%\textgreater{}\%}
  \FunctionTok{select}\NormalTok{(}\FunctionTok{c}\NormalTok{(month\_tsib, total\_lunches\_served)) }\SpecialCharTok{\%\textgreater{}\%} 
  \FunctionTok{as\_tsibble}\NormalTok{()}
\end{Highlighting}
\end{Shaded}

\begin{Shaded}
\begin{Highlighting}[]
\NormalTok{total\_decomp }\OtherTok{=}\NormalTok{ monthly\_tsib }\SpecialCharTok{\%\textgreater{}\%} 
  \FunctionTok{model}\NormalTok{(}
    \FunctionTok{classical\_decomposition}\NormalTok{(total\_lunches\_served, }\AttributeTok{type =} \StringTok{"additive"}\NormalTok{)}
\NormalTok{  ) }\SpecialCharTok{\%\textgreater{}\%} 
  \FunctionTok{components}\NormalTok{()}
\FunctionTok{head}\NormalTok{(total\_decomp)}
\end{Highlighting}
\end{Shaded}

\begin{verbatim}
## # A dable: 6 x 7 [1M]
## # Key:     .model [1]
## # :        total_lunches_served = trend + seasonal + random
##   .model         month_tsib total_lunches_s~ trend seasonal random season_adjust
##   <chr>               <mth>            <int> <dbl>    <dbl>  <dbl>         <dbl>
## 1 "classical_de~   2017 Oct        570181798    NA   1.52e8     NA    417772650.
## 2 "classical_de~   2017 Nov        494401221    NA   7.18e7     NA    422581769.
## 3 "classical_de~   2017 Dec        391520580    NA   2.35e7     NA    368067478.
## 4 "classical_de~   2018 Jan        472917391    NA   9.78e7     NA    375143033.
## 5 "classical_de~   2018 Feb        496059395    NA   9.53e7     NA    400760745.
## 6 "classical_de~   2018 Mar        491726562    NA   4.98e7     NA    441887118.
\end{verbatim}

\begin{Shaded}
\begin{Highlighting}[]
\FunctionTok{autoplot}\NormalTok{(total\_decomp)}
\end{Highlighting}
\end{Shaded}

\includegraphics{blog_draft_files/figure-latex/unnamed-chunk-9-1.pdf}

\begin{Shaded}
\begin{Highlighting}[]
\FunctionTok{acf}\NormalTok{(monthly\_tsib, }\AttributeTok{lag.max =} \DecValTok{12}\NormalTok{)}
\end{Highlighting}
\end{Shaded}

\includegraphics{blog_draft_files/figure-latex/unnamed-chunk-10-1.pdf}

\hypertarget{participation-of-free-lunch-eligible-students}{%
\subsection{Participation of Free-Lunch Eligible
Students}\label{participation-of-free-lunch-eligible-students}}

\begin{Shaded}
\begin{Highlighting}[]
\NormalTok{free\_tsib }\OtherTok{\textless{}{-}}\NormalTok{ usda\_monthly }\SpecialCharTok{\%\textgreater{}\%} 
  \FunctionTok{select}\NormalTok{(}\FunctionTok{c}\NormalTok{(month\_tsib, percent\_free\_of\_total\_lunches)) }\SpecialCharTok{\%\textgreater{}\%} 
  \FunctionTok{as\_tsibble}\NormalTok{()}
\end{Highlighting}
\end{Shaded}

\begin{Shaded}
\begin{Highlighting}[]
\NormalTok{free\_decomp }\OtherTok{=}\NormalTok{ free\_tsib }\SpecialCharTok{\%\textgreater{}\%} 
  \FunctionTok{model}\NormalTok{(}
    \FunctionTok{classical\_decomposition}\NormalTok{(percent\_free\_of\_total\_lunches, }\AttributeTok{type =} \StringTok{"additive"}\NormalTok{)}
\NormalTok{  ) }\SpecialCharTok{\%\textgreater{}\%} 
  \FunctionTok{components}\NormalTok{()}
\FunctionTok{head}\NormalTok{(free\_decomp)}
\end{Highlighting}
\end{Shaded}

\begin{verbatim}
## # A dable: 6 x 7 [1M]
## # Key:     .model [1]
## # :        percent_free_of_total_lunches = trend + seasonal + random
##   .model        month_tsib percent_free_of_~ trend seasonal random season_adjust
##   <chr>              <mth>             <dbl> <dbl>    <dbl>  <dbl>         <dbl>
## 1 "classical_d~   2017 Oct              68.8    NA    -2.67     NA          71.5
## 2 "classical_d~   2017 Nov              67.2    NA    -3.37     NA          70.6
## 3 "classical_d~   2017 Dec              67.2    NA    -3.89     NA          71.1
## 4 "classical_d~   2018 Jan              67.3    NA    -4.21     NA          71.5
## 5 "classical_d~   2018 Feb              68.1    NA    -4.22     NA          72.3
## 6 "classical_d~   2018 Mar              68.1    NA    -5.62     NA          73.7
\end{verbatim}

\begin{Shaded}
\begin{Highlighting}[]
\FunctionTok{autoplot}\NormalTok{(free\_decomp)}
\end{Highlighting}
\end{Shaded}

\includegraphics{blog_draft_files/figure-latex/unnamed-chunk-12-1.pdf}

\begin{Shaded}
\begin{Highlighting}[]
\FunctionTok{acf}\NormalTok{(free\_tsib, }\AttributeTok{lag.max =} \DecValTok{12}\NormalTok{)}
\end{Highlighting}
\end{Shaded}

\includegraphics{blog_draft_files/figure-latex/unnamed-chunk-13-1.pdf}

\hypertarget{summarize-results-visually-and-in-words}{%
\section{Summarize results visually and in
words}\label{summarize-results-visually-and-in-words}}

\hypertarget{next-steps-and-future-directions}{%
\section{Next steps and future
directions}\label{next-steps-and-future-directions}}

\hypertarget{references}{%
\section{References}\label{references}}

\begin{enumerate}
\def\labelenumi{(\arabic{enumi})}
\tightlist
\item
  \url{https://www.ers.usda.gov/topics/food-nutrition-assistance/child-nutrition-programs/national-school-lunch-program/}
\item
  \url{https://fns-prod.azureedge.net/sites/default/files/resource-files/NSLPFactSheet.pdf}
\end{enumerate}

\end{document}
